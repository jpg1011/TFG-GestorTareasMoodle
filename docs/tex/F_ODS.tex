\apendice{Anexo de sostenibilización curricular}

\section{Introducción}
En el desarrollo de este Trabajo de Fin de Grado he integrado los principios de sostenibilidad en la metodología de trabajo como en el producto final: una aplicación en Flutter con conexión a Moodle a través de su Web Service. Este trabajo ha reforzado mis competencias técnicas, y a su vez, también me ha servido para mejorar mi responsabilidad ética y mi visión estratégica para desarrollar soluciones tecnológicas que respeten los principios de los seres humanos y el entorno.

\section{Competencias de sostenibilidad adquiridas}
A lo largo de este proyecto he adquirido las siguientes nuevas competencias de sostenibilidad.

\subsection{Análisis crítico}
Capacidad de evaluación en relación a decisiones de diseño y arquitectura, enfocándose en los aspectos de eficiencia, accesibilidad, coste y beneficio a largo plazo.

\subsection{Responsabilidad social}
Priorización en la inclusión digital, asegurando que cualquier ser humano pueda usar la herramienta sin barreras de conocimiento.

\subsection{Ética profesional}
Gestión de los datos aplicando principios de transparencia y confidencialidad. Además, de proporcionar un documentación técnica para futuros desarrollos o revisiones.

\subsection{Gestión de recursos}
Optimización del tiempo, esfuerzo y energía de computación mediante el empleo de técnicas de desarrollo sostenible y automatización de procesos.

\section{Sostenibilidad ambiental}
Con el fin de reducir la huella ecológica del software se han seguido unas prácticas.

\subsection{Tecnología seleccionada}
Se eligió Flutter como tecnología de desarrollo porque tiene la capacidad de compilar a código nativo. Este proceso evita que las capas intermedias de tecnologías híbridas, reduciendo tiempos en el proceso de procesamiento, y por lo tanto, reduciendo la demanda energética.

\subsection{Estrategias de carga de datos bajo demanda}
La aplicación carga unicamente los datos del usuario al iniciar la sesión, y no permite efectuar una actualización de los mismos en ninguna otra pantalla. Esta estrategia evita transferencias de datos innecesarias.

\section{Sostenibilidad social y accesibilidad}
La aplicación refuerza la autonomía de la comunidad educativa al integrar herramientas que promueven la organización de tareas, así como facilitan la participación de todos los perfiles educativos que hagan uso de Moodle.

El diagrama de Gantt muestra de forma visual las actividades, ayudando a planificar y evitando solapamientos. La sección de tareas personales centraliza las responsabilidades académicas individuales en un único espacio, lo que reduce la dispersión de información. Además, el sistema de filtrado general permite al usuario filtrar contenidos, reduciendo la sobrecarga cognitiva y adaptándose a las necesidades de cada usuario.

\section{Sostenibilidad económica y mantenimiento}
Para garantizar la viabilidad del proyecto a largo plazo y optimizar los costes de su evolución, se ha aplicado un diseño modular, separando las distintas responsabilidades del sistema, lo que facilita la integración de nuevas funcionalidades. El código fuente se publica bajo los términos de una licencia libre, además la comunidad puede contribuir, corregir errores y reutilizar componentes para otros proyectos. De este modo, se minimizan los costes y se garantiza que la herramienta se mantenga segura, eficiente y alineada con las necesidades del usuario.

\section{Conclusión}
Este Trabajo de Fin de Grado me ha permitido integrar de forma realista los tres pilares sostenibilidad. He aprendido a optimizar el consumo de recursos evitando procesos innecesarios, a fomentar la inclusión de la comunidad educativa y a garantizar la continuidad y asequibilidad del proyecto. Este tiempo me ha brindado la lección, un enfoque que evalué cada decisión técnica en base a su impacto ambiental, social y económico, permite la creación de \textit{software} práctico y sostenible a largo plazo, y me prepara para afrontar con éxito los retos profesionales.
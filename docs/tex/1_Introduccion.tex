\capitulo{1}{Introducción}

Durante los últimos años la implementación de tecnologías en el ámbito educativo ha transformado los procedimientos de enseñanza. Las plataformas de gestión de aprendizaje (Learning Management Systems o LMS) como Moodle se han consolidado como herramientas imprescindibles en la enseñanza y en el seguimiento del progreso académico, ya que facilitan la interacción entre alumnos y docentes, organización del contenido académico, evaluación de actividades, etc. Aunque el uso de Moodle es cómodo para el alumnado, cabe recalcar que carece de algunas funcionalidades que pueden enriquecer el uso de la plataforma.

La motivación que impulsa el desarrollo de este proyecto viene de mejorar la gestión de las actividades de Moodle, para dar una visión más clara al alumnado de las tareas que debe realizar, previniendo que las labores pendientes queden desatendidas. Además, la aplicación incorpora un sistema de filtrado que proporciona un abanico de amplias posibilidades en el momento de filtrar el contenido que el usuario desee.

Por ello, en este Trabajo de Fin de Grado se propone la implementación de una aplicación multiplataforma desarrollada con Dart y Flutter, con el objetivo de complementar y mejorar la experiencia que ofrece Moodle. Para lograr esto último es necesario recurrir a los servicios web ofrecidos por Moodle, lo que permite la autenticación de usuarios, consulta de cursos, actividades a realizar o calificaciones obtenidas, entre decenas de funciones. Por otro lado, se presta especial atención al diseño de la interfaz de usuario, así como a la experiencia de usuario, mejorando la presentación y el uso de la aplicación.

\subsection{Estructura del material entregado}
A continuación se detallarán las estructuras de los materiales entregados. Desde la memoria y anexos, hasta la estructura de directorios y ficheros alojados en el repositorio de GitHub.

\subsubsection{Estructura de la memoria}

\begin{itemize}
    \item \textbf{Introducción:} proporciona una visión general del contexto del proyecto, además de la estructura que lo compone.
    \item \textbf{Objetivos del proyecto:} describe los objetivos que se persiguen para alcanzar las metas establecidas.
    \item \textbf{Conceptos teóricos:} explica conceptos teóricos necesarios para la compresión total del proyecto y los objetivos que persigue.
    \item \textbf{Técnicas y herramientas:} describe las técnicas y herramientas empleadas para lograr el desarrollo del proyecto.
    \item \textbf{Aspectos relevantes del desarrollo del proyecto:} detalla los aspectos más destacados durante el desarrollo del proyecto, desde los inconvenientes hasta las decisiones tomadas para resolver los desafíos del proyecto.
    \item \textbf{Trabajos relacionados:} revisión de proyectos similares, realizando comparativas con dichos proyectos.
    \item \textbf{Conclusiones y líneas de trabajo futuras:} recoge las conclusiones finales del proyecto y especifica posibles mejoras a implementar en el futuro.
\end{itemize}

\subsubsection{Estructura de los anexos}

\begin{itemize}
    \item \textbf{Plan de proyecto software:} detalla la planificación del proyecto y estudia las viabilidades económicas y legales del mismo.
    \item \textbf{Especificación de requisitos:} detalla los requisitos y casos de uso del proyecto.
    \item \textbf{Especificación de diseño:} especifica los aspectos de diseño, detallando la interfaz de usuario, diseño de datos, diagrama de clases, etc.
    \item \textbf{Documentación técnica de programación:} aporta documentación con la explicación de aspectos técnicos del proyecto para programadores.
    \item \textbf{Documentación de usuario:} aporta documentación relevante para que el usuario tenga conocimiento sobre el uso de la aplicación.
\end{itemize}

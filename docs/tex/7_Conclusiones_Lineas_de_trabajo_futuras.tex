\capitulo{7}{Conclusiones y Líneas de trabajo futuras}

\section{Conclusiones}
Una vez finalizado el desarrollo del proyecto de la aplicación multiplataforma para gestionar tareas de Moodle, puedo afirmar que se han cumplido con todos los objetivos propuestos y, además, se han incorporado funcionalidades que no estaban previstas desde comienzo del proyecto. Por lo tanto, se han completado con todos los objetivos del proyecto. El hecho de haber alcanzado las expectativas previstas supone un logro gratificante, que es fruto de una planificación en la que se realizaban constantes revisiones así como nuevas propuestas.

En el aspecto personal, me siento satisfecho con el resultado final, ya que se ofrece una aplicación al alumnado que puede usar de forma que le sirva para la planificación de su tiempo en el momento de realizar sus tareas. El hecho de haber resuelto todos los obstáculos presentes durante la realización del proyecto, ha supuesto un logro personal, donde me he demostrado a mi mismo que con constancia todo es posible.

Por otra parte, ha sido positivo adquirir nuevos conocimientos relacionados con el desarrollo de aplicaciones multiplataforma. Además, me ha motivado a seguir ampliando mis conocimientos acerca de esta tecnología. Estos conocimientos podrán ser utilizados en el futuro para el desarrollo de nuevos proyectos o la continuación de este mismo si se diera la oportunidad.

\section{Líneas de trabajo futuras}
En esta sección se expondrán las posibles futuras implementaciones que podrían enriquecer el uso de la aplicación así como atraer a nuevos usuarios.

\subsection{Resumen diario}
Se podría incorporar una nueva funcionalidad que informe al usuario sobre su estado en la plataforma, este resumen le brindaría al usuario un resumen diario con información como:
\begin{itemize}
    \item Tareas pendientes para el día actual o los siguientes días.
    \item Nuevas calificaciones en tareas o cuestionarios.
\end{itemize}
Esta funcionalidad nos proporcionaría un enfoque de lo útil que sería usar esta aplicación, ya que el usuario estaría en todo momento informado de sus labores o resultados de las mismas.

\subsection{Visualización de estadísticas}
El hecho de permitir que los usuarios visualicen el progreso en sus tareas o cursos puede servir de ayuda de cara al usuario para concienciarlo en el cumplimiento de sus actividades. Es decir, si un usuario no esta dedicando mucho tiempo a un curso, esta funcionalidad le puede incentivar a ser mas constante y atento. 

Dentro de esta funcionalidad se pueden incluir muchos tipos de estadísticas, estas pueden ser:
\begin{itemize}
    \item Porcentaje de actividades completadas respecto del total.
    \item Tiempo medio dedicado en la realización de actividades.
    \item Media ponderada de las calificaciones de actividades.
\end{itemize}

\subsection{Internacionalización a otro idiomas}
Actualmente la aplicación se encuentra en castellano, por lo que limita su uso a las regiones donde se haga uso de este idioma. El hecho de expandirse por otros horizontes requiere de adaptarse a las necesidades de los usuarios. Es por ello, que uno de los puntos cruciales para poder llegar a más público, es la integración de nuevos idiomas en la aplicación. Como consecuencia se podría comercializar a nivel global.

\subsection{Implementación nuevos diagramas}
Para mejorar la experiencia de gestión de tareas, se propone la integración de nuevos diagramas, gráficos o formas de representación. El hecho de recurrir a nuevas formas de representación de tareas, puede proporcionar al usuario distintas perspectivas para la gestión y planificación de sus labores. Los gráficos a implementar podrían ser:
\begin{itemize}
    \item \textbf{Tablero Kanban:} consiste en un tablero divido en distintas secciones: ``Por hacer``, ``En progreso`` y ``Hecho``, entre otras varias posibilidades. Esta forma de trabajo permite desplazar las tareas entre las distintas secciones dependiendo de su estado.
    \item \textbf{Calendario o agenda:} herramientas que permiten asignar fechas a las tareas con la finalidad de programar plazos o bloques de trabajo.
    \item \textbf{Checklist:} se trata de una lista simple de tareas donde las realizadas quedan marcadas con una casilla, indicando su estado de finalización.
\end{itemize}

\subsection{Sistema de notificaciones}
Un aspecto muy importante a implementar de cara al futuro es la notificación de los eventos a realizar. Es muy importante que el usuario reciba avisos para no perder la constancia en la realización de sus labores. Este sistema se podría plantear de forma que el usuario pueda configurar las notificaciones según sus necesidades o intereses.

\subsection{Implementación de nuevos filtros}
El hecho de que la aplicación se pueda expandir hacia todos los horizontes, hace que el sistema de filtrado cobre mayor importancia. Por ello, la implementación de nuevos filtros podría suponer una interfaz más limpia así como una organización potenciada por dichos filtros. Los filtros que podrían ser implementados son:
\begin{itemize}
    \item Filtro por estado de entrega.
    \item Filtro por prioridad de la tarea (Tareas personales).
    \item Filtro por calificaciones.
    \item Filtro por duración de las tareas.
\end{itemize}

\subsection{Integración con otras plataformas}
Actualmente la aplicación se conecta con las plataformas Moodle, función que permite internacionalizar la aplicación en distintas instituciones que utilicen Moodle. Para llegar a más usuarios, se plantea implementar la compatibilidad con otras plataformas de gestión de tareas y cursos. Algunas plataformas son:
\begin{itemize}
    \item Blackbox
    \item Canvas
    \item Totara
\end{itemize}

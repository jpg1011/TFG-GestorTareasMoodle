\capitulo{2}{Objetivos del proyecto}

El objetivo principal del proyecto es el desarrollo de una aplicación multiplataforma con Dart y Flutter, que permita al alumnado de distintas instituciones educativas planificar las actividades de su plataforma Moodle. Se busca potenciar la organización del usuario para incrementar su rendimiento académico, así como sus capacidades de planificación de tareas.

El proyecto se centra en uso de un diagrama de Gantt que facilita la visualización temporal de las actividades a realizar. Por otra parte, se implementa la creación de tareas con niveles de prioridad y ligadas a un curso matriculado. Estas funcionalidades pueden ser filtradas al gusto del usuario con el sistema de filtrado incorporado.

\begin{itemize}
    \item \textbf{Visualización gráfica de actividades:} El objetivo principal es la implementación de un diagrama de Gantt interactivo que permita al usuario visualizar de forma clara y ordenada las fechas y duración de las tareas, de las actividades de Moodle. Además, el usuario podrá desplazarse a través del diagrama y podrá ver sus actividades ordenadas por cursos en un sección pareja al diagrama.
    \item \textbf{Creación de tareas propias:} La aplicación ofrecerá al usuario la posibilidad de definir sus propias tareas ligadas a un curso en el que se encuentre matriculado y podrá asignarles unas prioridad entre otras opciones. Todas las tareas creadas por parte del usuario se mostrarán por pantalla en función de su estado: Pendiente o Completada. Además, estarán ordenadas en función de su fecha de caducidad o su fecha de finalización.
    \item \textbf{Filtrado general de actividades y tareas:} Una de las funcionalidades más importantes será el filtrado de todos los eventos de ambas funcionalidades mencionadas anteriormente. Esto será posible gracias al sistema de filtrado disponible en la pantalla inicial. Desde este, el usuario podrá filtrar por cursos, fechas, tipos de actividad y disponibilidad de fechas de apertura y cierre. Con ello se pretende que el usuario modifique el contenido de su aplicación a gusto propio y así logre mejorar su capacidad de organización.
\end{itemize}

\subsection{Objetivos personales}
En esta sección se establecen los objetivos personales que se pretenden superar en la realización de este proyecto:

\begin{itemize}
    \item Aprendizaje en el desarrollo de aplicaciones, en este caso multiplataforma, para futuros proyectos.
    \item Mejorar el manejo de APIs.
    \item Entender el uso de la metodología SCRUM para su uso aplicado en proyectos reales.
    \item Mejorar la lógica de programación para desarrollos complejos.
    \item Descubrir y explorar nuevas herramientas que no había empleado con anterioridad.
\end{itemize}

\subsection{Objetivos técnicos}
En esta sección se especifican los objetivos técnicos que se pretenden alcanzar con la realización del proyecto:

\begin{itemize}
    \item Uso de GitHub con GitBash para el control de versiones.
    \item Uso de Visual Studio Code para el desarrollo del proyecto.
    \item Uso de Dart y Flutter para el desarrollo de la aplicación multiplataforma.
    \item Uso de Postman para realizar peticiones a la API.
    \item Uso de AndroidEmulator para la simulación de un entorno Android.
    \item Uso de Figma para el prototipo de la aplicación.
    \item Uso de Overleaf para la realización de la documentación.
\end{itemize}